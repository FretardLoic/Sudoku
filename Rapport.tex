\documentclass[a4paper,10pt]{report}
\usepackage[utf8]{inputenc}
\usepackage[T1]{fontenc}
\usepackage[francais]{babel}
\usepackage{hyperref}
\usepackage{attrib}

% Title Page
\title{Rapport de Projet de fin de Licence : Sudoku}
\author{Frétard Boucher Brun Hardouin}

\begin{document}
\maketitle

\chapter*{Définitions d'un Sudoku}

\begin{quote}
 La grille de jeu est un carré de neuf cases de côté,
 subdivisé en autant de carrés identiques, appelés régions.
 La règle du jeu est simple : chaque ligne, colonne et région ne doit
 contenir qu'une seule fois tous les chiffres de un à neuf.
 Formulé autrement, chacun de ces ensembles doit contenir tous
 les chiffres de un à neuf.

 \attrib{\href{http://www.lesudokugratuit.com/fr/article/description/17/definition_officielle_du_sudoku}
 {lesudokugratuit.com}}
\end{quote}

\begin{quote}
 Un Sudoku est un casse-tête dans lequel des nombres doivent remplir une grille
 de 9 par 9 carrés subdivisés en régions de 3 par 3 de telle sorte que toute ligne,
 toute colonne, et toute région contient les nombres de 1 à 9.
 
 \attrib{\href{http://www.merriam-webster.com/dictionary/sudoku}{merriam-webster.com}
  {\em traduction}}
\end{quote}

\chapter*{Fonctionnalités}

% definition des besoins
% 	diagramme de cas d'utilisations
% 


\section*{Modèle}

\begin{itemize}
 \item prise en charge de grilles de dimensions variables
 \item fichier de chargement pour une nouvelle grille
 \item sauvegarde/chargement d'une partie
 \item réinitialisation de la grille
 \item vérification de la victoire automatique
 \item vérification des conflits à la demande de l'utilisateur
 \item modification d'une cellule
 \begin{itemize}
  \item valeur définitive
  \begin{itemize}
   \item élimination des possibilité pour la valeur sur
   la ligne, colonne et région sur lesquels se trouve la cellule
  \end{itemize}
  \item possibilités
  \item effacement
 \end{itemize}
 \item demander l'aide
 \begin{itemize}
  \item expliquer la règle
  \item appliquer la règle
 \end{itemize}
 \item résoudre la grille pas à pas
 \item résoudre complètement
 \begin{itemize}
  \item si la grille ne peut pas être résolue,
  proposer une réinitialisation avant résolution
 \end{itemize}
 \item classement par dificulté (en fonction des heuristiques nécessaires)
 \item timer - Optionnel
 \item pause - Optionnel
 \item compteur de coups - Optionnel
 \item éditeur/générateur (par difficulté?) - Optionnel
 \item samourai - Optionnel
\end{itemize}

\section*{Vue}

\begin{itemize}
 \item raccourcis clavier
 \item modification d'une cellule
 \begin{itemize}
  \item clic droit :
  \begin{itemize}
   \item sur une possibilité, met la valeur de la cellule à cette possibilité
   \item sur une valeur, efface la valeur et affiche les possibilités
  \end{itemize}
  \item clic gauche :
  \begin{itemize}
   \item sur une possibilité, active/desactive la possibilité
   \item sur une valeur, rien
  \end{itemize}
 \end{itemize}
 \item demande d'aide
 \begin{itemize}
  \item mettre en surbrillance la/les case(s) concernées
 \end{itemize}
 \item surface de la grille fixe
 \item tutoriel - Optionnel
 \item masquer la grille pendant la pause - Optionnel
 
\end{itemize}

\end{document}          
