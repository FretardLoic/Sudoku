\section*{Introduction} % Pas de numérotation
\addcontentsline{toc}{section}{Introduction} % Ajout dans la table des matières

Ce projet a pour objectif de réaliser un lanceur de commandes.
L'utilisateur pourra ainsi sélectionner le programme et exécuter 
tous les commandes du système Unix exactement de la même façon 
que le ferait un terminal.
Les commandes pourront être accompagnés d'options qui seront gérées également.
% \vspace{\parskip}
\smallskip

Le projet est ainsi constitué de deux programmes :
\begin{itemize} % liste classique
\item un démon qui lancera les commandes ;
\item un client capable de transmettre les commandes à exécuter au démon.
\end{itemize}
%\vspace{\parskip} % espace entre paragraphes
\smallskip

De plus, la communication entre démon et client se fait 
obligatoirement selon le schéma suivant :
\begin{itemize}
 \item les clients transmettent leurs requètes 
       grâce à une file synchronisée implémentée 
       à l'aide d'un segment de mémoire partagé ;
 \item la requète du client sera constitué 
       de la commande à exécuter et de ses 
       éventuels arguments et le nom de deux 
       tubes nommés pipe1 et tube2 ;
 \item la commande sera exécutée par 
       le démon qui aura redirigé 
       l'entrée standard de celle-ci 
       vers pipe1 et 
       la sortie standard vers tube2 ;  
 \item le client transmettra alors à 
       la commmande toutes les données 
       lues sur son entrée standard 
       grâce à pipe1 et il affichera 
       toutes les données reçues sur tube2
       sur sa sortie standard ;
 \item toutes les requètes reçues par 
       le démon seront traitées par des threads dédiés.
\end{itemize}
\vspace{\parskip}

Le démon devra gérer correctement tout phénomène de zombies 
ainsi que toutes demandes de terminaisons via des signaux.

Chaque acteur devra libérer proprement les ressources utilisées.

Le segment de mémoire partagé où seront stockés 
les requètes des clients devra pouvoir contenir plusieurs requètes 
et sont fonctionnement sera de type FIFO, 
les requètes seront récupérées par le démon 
dans l'ordre où elles ont été stockées 
dans le segment de mémoire par le client.

