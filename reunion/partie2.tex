\definecolor{Zgris}{rgb}{0.87,0.85,0.85}

\newsavebox{\BBbox}
\newenvironment{DDbox}[1]{
\begin{lrbox}{\BBbox}\begin{minipage}{\linewidth}}
{\end{minipage}\end{lrbox}\noindent\colorbox{Zgris}{\usebox{\BBbox}} \\
[.5cm]}

\section{manuel technique}

\subsection{Mode de communication}

Pour développer notre lanceur de commandes, 
nous avons créer une structure que 
nous partageons en mémoire partagée :


\begin{DDbox}{\linewidth}
  \begin{Verbatim}
      //structure des informations du clients
      struct dd {
	      char pipe[PIPE_SIZE + 1];
	      char tube[TUBE_SIZE + 1];
	      char requete[REQUETE_SIZE + 1];
      };
      typedef struct dd dd;

      // Nos variables en mémoire partagée.
      struct fifo {
        sem_t numero;
	sem_t mutex;
	sem_t vide;
	sem_t plein;
	// Position d'ajout dans le tampon
	int tete;      
	// Position de suppression dans le tampon
	int queue;    
	// La structure contenant les informations des N clients
	dd buffer[N]; 
      };

      #define TAILLE_SHM (sizeof(struct fifo))
  \end{Verbatim}
\end{DDbox}

\newpage

Pour cela, il nous a fallu définir les noms 
et les tailles des différents éléments utilisés :

\begin{DDbox}{\linewidth}
  \begin{Verbatim}
    #define TAILLE_SHM (sizeof(struct fifo))

    /**
    * Un nom de segment de mémoire partagé 
    * avec un identifiant pour nous assurer de son unicité.
    */
    #define NOM_SHM "/mon_shm_a_moi_58942298536542745545"
    #define NOM_FIFO_ENTRANT "mon_tube_nomme_1488864545622"
    #define NOM_FIFO_SORTANT "mon_tube_nomme_1488862545623"

    // Taille limite du nombre de clients en même temps
    #define N 10
    #define PIPE_SIZE 1024
    #define TUBE_SIZE 1024
    #define REQUETE_SIZE 1024
    #define RESULTAT_SIZE 1024
  \end{Verbatim}
\end{DDbox}


\subsection{client}
Ce programme gère les demandes des clients, 
il s'occupe d'ajouter les requètes des clients dans la mémoire partagée.

\subsection{demon}
Ce programme, lancé en arrière plan, tourne toujours 
et gère les requètes des clients récupérées depuis la mémoire partagée.  

\newpage

\subsection{Difficultés rencontrées}
Au cours de l'implantation de ce projet, 
nous avons rencontré quelques difficultés :

\begin{enumerate}
 \item la prise de connaissance du sujet, 
 où nous avons du gérer des critères spécifiques (utilisation de tubes, de pipes etc.)
 \item le choix des structures pour le partage de mémoire,
 \item la création du démon, afin que ce dernier reste en arrière plan,
 \item la communication entre les différents acteurs, ou comment gérer 
 la communication entre un démon et un ou plusieurs clients,
 \item la gestion de plusieurs clients,
 \item les signaux,
 \item le temps imparti du projet. 
\end{enumerate}
